\section{figure\_section\_1l}

This function permits to plot 3 panels, with temperature, salinity and
velocity, respectively. The function is called as follows:

\begin{verbatim}
pypago.figure_section_1l('structfile',secplot='ar7',fname=None)
\end{verbatim}

The first argument, which is compulsory, is the structure file that
contains the variables to plot (computed using the loaddata
functions). The two other arguments are optional. The \verb+secplot+
argument is the name of the section, whose fields to plot. If the
default value is used (\verb+None+), the list of available sections is
prompted and the user choses which to use. The
\verb+fname+ argument is the name of the file where to save the figure
. If the default \verb+None+ value is used, the user is asked whether
save or not the figure. If so, then the user is asked to write down
the name of the output file.\\

The program is rather interactive. Below is provided an example of the
commands entered by the user:

\begin{verbatim}
Ok with the color axis of temperature? 0
Define new color axis: cmin, cmax (current: -1.62863957882 5.57030677795) -2,6
Ok with the color axis of temperature? 1
Levels for the contour lines of temperature? np.arange(-2,6.5,0.5)
Ok with the stride of the temperature colorbar labels? 0
Give colorbar strides (integer) 2
Ok with the stride of the temperature colorbar labels? 1

Ok with the color axis of salinity? 1
Levels for the contour lines of salinity? [32.4,32.8,34,34.4]
Ok with the stride of the salinity colorbar labels? 1
Ok with the color axis of velocity? 0
Define new y axis: cmin, cmax (current: -52.1449410739 69.2010911407) -50,50

Ok with the color axis of velocity? 1
Levels for the contour lines of velocity? []
Ok with the stride of the velocity colorbar labels? 1
Ok with the y axis? 0
Define new y axis: ymin, ymax (current: -3.64657297564 0.0) -3,0

Ok with the y axis? 1
Ok with the x axis? 1
Save the figure (0/1) ? 1
Name of the figure? toto.pdf
\end{verbatim}

The resulting figure is depicted in figure \ref{fig:figure_section}.

\begin{figure}[h!]
\centering
\includegraphics[scale=0.52]{toto.pdf}
\caption{Figure obtained using the figure\_section\_1l function.}
\label{fig:figure_section}
\end{figure}

If only one variable is to be plotted, consider using the
\verb+pypago.figure_section_1l_1var+ program instead, which takes as an
additional argument, a strng that contains the name of the field.