\section{ihm\_grid\_model.py}

When the section file has been created, the next step is to determine
the section indexes in the grid coordinates of the model of
interest. This is done using the \verb+ihm_grid_model.py+ program,
whose interface is depicted in figure \ref{fig:grid_1}.

A video showing how to use the program is shown \href{https://dl.dropboxusercontent.com/u/99128427/Site/barriernicolas/Python_NCL_files/video_ihm_grid_model.mov}{here}.

\begin{figure}[h!]
\centering
\includegraphics[scale=0.5]{figs/grid_1.png}
\caption{GUI of the ihm\_grid\_model.py program}
\label{fig:grid_1}
\end{figure}

\subsection{Initialisation}

In order to initialise the program, one must chose which model is
being processed. This is done by changing the model name in the
ComboBox \verb+Model+. When the model is chose, one has to open the
model mesh file containing all the information needed by PyPago. This
is done by using the \verb+File -> Open+ menu.\\

This automatically draws the model coastlines in grid coordinates, as
depicted in figure \ref{fig:grid_2}.

\begin{figure}[h!]
\centering
\includegraphics[scale=0.5]{figs/grid_2.png}
\caption{GUI when a mesh file is opened (here, the mesh file of the
  CNRM-CM5 model.}
\label{fig:grid_2}
\end{figure}

The black lines and the orange dots depict the domain, on which the
data will be extracted. By default, the greatest domain is selected.

\subsection{Domain selection}

To change the domain, several possibilities exist. First, the user can
define the domain by editing the text control boxes (\verb+min_i+ and
\verb+max_i+ for the x-limits, \verb+min_j+ and
\verb+max_j+ for the y-limits). The values are taken into account only
when the ENTER key is pressed. This way of changing the coordinates is
the only possibility to define disconstinued domain, where
\verb+min_i>max_i+, as shown in figure \ref{fig:grid_3}.


\begin{figure}[htp]
  \centering
  \subfloat[min\_i < max\_i]{\label{fig:grid_3_a}\includegraphics[scale=0.3]{figs/grid_3a.png}}
  \subfloat[min\_i > max\_i]{\label{fig:grid_3_b}\includegraphics[scale=0.3]{figs/grid_3b.png}}
  \caption{Example of domain modification.}
  \label{fig:grid_3}
  \label{fig:contour}
\end{figure}

Another possibility to edit the domain is by changing the coordinates
of the corners. To do so, click on one of the orange point and move
the mouse. The new coordinates are validated when the mouse is
released.

\subsection{Computation of section indices}

When the proper domain is selected, one needs to load a section file
(as generated by the \verb+ihm_editions_sections.py+ program). This is
done by using the \verb+Section -> Load section+ menu. When the file
is opened, a new figure is drawn, in which the section appears in
model coordinates, as show in figure \ref{fig:grid_4}. If a section is located out of the domain, a dialo box is opened,
warning the user that some sections, which are listed, have been discarded.

\begin{figure}[h!]
\centering
\includegraphics[scale=0.5]{figs/grid_4.png}
\caption{Output when a section file is edited.}
\label{fig:grid_4}
\end{figure}

In figure \ref{fig:grid_4}, we notice that we have several problems. First, the
lefternmost segment is not well oriented, since all the dots are not
located at the same side of the line. Secondly, the last point of the
last segment is in the water, which might be a problem is computing
area indices. In order to correct these issues, the user may edit the
sections by selecting the \verb+Edit sections+ item of the
radiobox. 

\subsection{Section editions}

The user can modify the orientation of the segments and the
position of the section corners (orange points), in the same way as in the
\verb+ihm_editions_sections.py+ program (figure \ref{fig:grid_5}).

\begin{figure}[h!]
\centering
\includegraphics[scale=0.5]{figs/grid_5.png}
\caption{GUI when the sections are edited.}
\label{fig:grid_5}
\end{figure}

\subsection{Output}

In order to save the outputs of the program, the user must call the
menu \verb+File -> Save+ or \verb+File -> Save as+. Note that in order
to persuade the user to check that the sections have been well
defined, these two menus are only available if the 
\verb+Check sections+ item of the radiobox is selected.