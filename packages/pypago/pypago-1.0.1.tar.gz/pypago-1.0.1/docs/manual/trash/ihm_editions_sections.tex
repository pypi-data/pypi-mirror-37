\section{ihm\_editions\_sections.py}

Prior to using PyPago, one needs to define section endpoints. This is
achieved by running the \verb+ihm_editions_sections.py+ program, whose
GUI is as shown in figure \ref{fig:sections_1}.\\

\begin{figure}[h!]
\centering
\includegraphics[scale=0.5]{figs/sections_1.png}
\caption{GUI of the ihm\_editions\_sections.py program}
\label{fig:sections_1}
\end{figure}

A video showing how to use this program is available
\href{https://dl.dropboxusercontent.com/u/99128427/Site/barriernicolas/Python_NCL_files/video_section_def.mov}{here}.\\

\subsection{Map handling}

In the upper-left corner, the widgets are intented to map handling (\verb+Map handling+ box):
\begin{enumerate}
\item The \verb+lone+, \verb+lonw+, \verb+lats+, \verb+latn+ are
  the geographical limits of the map (eastern and western longitudes,
  southern and northern latitudes respectively). Default is a global map.
\item The \verb+lon0+ and \verb+boundinglat+ are the values of the
  central longitude and of the bounding latitude for pole-centered projections.
\item The projection of the map. By default, the projection is {Cylindrical Equidistant}
(\verb+cyl+). The available projections are {North-Polar
  Stereographic} (\verb+npstere+), {South-Polar Stereographic}
(\verb+spstere+) and {Lambert Conformal} (\verb+lcc+).
\item The resolution of the map: crude (\verb+c+), low (\verb+l+),
  intermediate (\verb+i+), high (\verb+h+) and full (\verb+f+). The
  full resolution must only be used for very localised studies (it is
  very slow otherwise)
\item A checkbox that specifies the map background. If the box is
  checked, the Etopo bathymetry is drawn using the
  \verb+basemap.etopo()+ function (in this case, the resolution option
  is useless). Else, filled continents are used instead.
\end{enumerate}

To modify the domain boundaries, one has to edit the text in the text
controls. The modifications are taken into account only when the ENTER
key is pressed.

\subsection{Section definition}

When the proper map layout is selected, the user may either define new
sections or edit new sections. The modes are activated using the radiobox \verb+Choose the edition mode+. 

\subsubsection{New sections}

If the radiobox is on the \verb+Add sections+ item, the user may define
new sections as follows.\\

This is done by clicking on the map, in order to add section
endpoints, as shown in figure \ref{fig:sections_2}.\\

\begin{figure}[h!]
\centering
\includegraphics[scale=0.5]{figs/sections_2.png}
\caption{Example of section definition.}
\label{fig:sections_2}
\end{figure}

When the section is
completed, it must be validated by a right-click. Then, another
section can be added.\\ By default, the section are named
\verb+section1+, \verb+section2+, \verb+section3+ and so on.\\

When all the sections have been defined and validated, they must be edited.

\subsubsection{Section edition}

If the radiobox is on the \verb+Edit sections+ item, the user may edit
the existing sections.\\

First, the user must chose a section to edit. This is achieved by
clicking on one of the line. When done, the sections corners appear on
orange, the name of the section is given in a text control box and the
different segments are named, as
shown in figure \ref{fig:sections_3}.\\ 

Then the user can
change the name of the section. This is done by modifing the text in
the text control box and by pressing the ENTER key. 
The position of the section corners can also be modified. This is
done by clicking on one of the orange points and dragging the mouse
into the new coordinates. These new coordinates are implemented when
the mouse is released. If the entire section, and not only one point,
are to be modified, the same operation must be repeated but by clicking
on the line first.
The user may also delete the selected section by clicking on the
\verb+Delete+ button.\\

\begin{figure}[h!]
\centering
\includegraphics[scale=0.5]{figs/sections_3.png}
\caption{Example of section edition.}
\label{fig:sections_3}
\end{figure}

In order to run PyPago properly, the user must carefully define the
orientations of each segment (all the sections are North-Eastward by default). This is achieved by clicking on one of
the segment box, as shown in figure \ref{fig:sections_4}. Then, the
user is able to change the orientation of the section: North-East
(\verb+NE+), North-West (\verb+NW+), South-East (\verb+SE+) and South-West (\verb+SW+).\\

\begin{figure}[h!]
\centering
\includegraphics[scale=0.5]{figs/sections_4.png}
\caption{Example of segment edition.}
\label{fig:sections_4}
\end{figure}

\subsection{Input/Output}

When done, the user is able to save the sections.  This is done using
the \verb+File+ menu, which contains a \verb+Save+ and a \verb+Save as+
item. The user should specify the file extension that is used for all
PyPago related files, which has been chosen to be \verb+.pygo+. The
user can also load an existing file using the \verb+Open+ item of the
\verb+File+ menu.
