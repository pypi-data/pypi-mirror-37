\chapter{Model Building as Software Development}

\noindent
Developing a statistical model in Stan means writing a Stan program
and is thus a kind of software development process.  Developing
software is hard.  Very hard.  So many things can go wrong because
there are so many moving parts and combinations of parts.

Software development practices are designed to mitigate the problems
caused by the inherent complexity of writing computer programs.
Unfortunately, many methodologies veer off into dogma, bean counting,
or both.  A couple we can recommend that provide solid, practical
advice for developers are \citep{HuntThomas:99} and
\citep{McConnell:2004}.  This section tries to summarize some of their
advice.

\section{Use Version Control}

Version control software, such as Subversion or Git, should be in
place before starting to code.%
%
\footnote{Stan started using Subversion (SVN), then switched to the
  much more feature-rich Git package.  Git does everything SVN does
  and a whole lot more.  The price is a steeper learning curve.  For
  individual or very-small-team development, SVN is just fine.}
%
It may seem like a big investment to learn version control, but it's
well worth it to be able to type a single command to revert to a
previously working version or to get the difference between the
current version and an old version.  It's even better when you need
to share work with others, even on a paper---work can be done
independently and then automatically merged.  See
\refchapter{software-development} for information on how Stan itself
is developed.


\section{Make it Reproducible}

Rather than entering commands on the command-line when running models
(or entering commands directly into an interactive programming
language like R or Python), try writing scripts to run the data
through the models and produce whatever posterior analysis you need.
Scripts can be written for the shell, R, or Python.  Whatever language
a script is in, it should be self contained and not depend on global
variables having been set, other data being read in, etc.

See \refchapter{reproducibility} for complete information on
reproducibility in Stan and its interfaces.

\subsection{Scripts are Good Documentation}

It may seem like overkill if running the project is only a single line
of code, but the script provides not only a way to run the code, but
also a form of concrete documentation for what is run.


\subsection{Randomization and Saving Seeds}

Randomness defeats reproducibility.  MCMC methods are conceptually
randomized.  Stan's samplers involve random initializations as well as
randomization during each iteration (e.g., Hamiltonian Monte Carlo
generates a random momentum in each iteration).

Computers are deterministic.  There is no real randomness, just
pseudo-random number generators.  These operate by generating a
sequence of random numbers based on a ``seed.''  Stan (and other
languages like R) can use time-based methods to generate a seed based
on the time and date, or seeds can be provided to Stan (or R) in the
form of integers.  Stan writes out the seed used to generate the
data as well as the version number of the Stan software so that
results can be reproduced at a later date.%
%
\footnote{This also requires fixing compilers and hardware, because
  floating-point arithmetic does not have an absolutely fixed behavior
  across operating systems, hardware configurations, or compilers.}



\section{Make it Readable}

Treating programs and scripts like other forms of writing for an
audience provides an important perspective on how the code will be
used.  Not only might others want to read a program or model, the
developer will want to read it later.  One of the motivations of
Stan's design was to make models self-documenting in terms of variable
usage (e.g., data versus parameter), types (e.g., covariance matrix
vs. unconstrained matrix) and sizes.

A large part of readability is consistency.  Particularly in naming
and layout.  Not only of programs themselves, but the directories and
files in which they're stored.  Readability of code is not just about
comments---it is also about naming and organization for readability.

It is surprising how often the solution to a debugging or design
problem occurs when trying to explain enough about the problem to
someone else to get help.  This can be on a mailing list, but it works
best person-to-person.  Finding the solution to your own problem when
explaining it to someone else happens so frequently in software
development that the listener is called a ``rubber ducky,'' because
they only have to nod along.%
%
\footnote{Research has shown an actual rubber ducky won't work.  For
  some reason, the rubber ducky must actually be capable of
  understanding the explanation.}


\section{Explore the Data}

Although this should go without saying, don't just fit data blindly.
Look at the data you actually have to understand its properties.  If
you're doing a logistic regression, is it separable?  If you're
building a multilevel model, do the basic outcomes vary by level?  If
you're fitting a linear regression, see whether such a model makes
sense by scatterplotting $x$ vs. $y$.

\section{Design Top-Down, Code Bottom-Up}

Software projects are almost always designed top-down from one or more
intended use cases.  Good software coding, on the other hand, is
typically done bottom-up.

The motivation for top-down design is obvious.  The motivation for
bottom-up development is that it is much easier to develop software
using components that have been thoroughly tested.  Although Stan has
no built-in support for either modularity or testing, many of the same
principles apply.

The way the developers of Stan themselves build models is to start as
simply as possibly, then build up. This is true even if we have a
complicated model in mind as the end goal, and even if we have a very
good idea of the model we eventually want to fit.  Rather than
building a hierarchical model with multiple interactions, covariance
priors, or other complicated structure, start simple.  Build just a
simple regression with fixed (and fairly tight) priors.  Then add
interactions or additional levels.  One at a time.  Make sure that
these do the right thing.  Then expand.

\section{Fit Simulated Data}

One of the best ways to make sure your model is doing the right thing
computationally is to generate simulated (i.e., ``fake'') data with
known parameter values, then see if the model can recover these
parameters from the data.  If not, there is very little hope that it
will do the right thing with data from the wild.

There are fancier ways to do this, where you can do things like run
$\chi^2$ tests on marginal statistics or follow the paradigm
introduced in \citep{CookGelmanRubin:2006}, which involves interval
tests.

\section{Debug by Print}

Although Stan does not have a stepwise debugger or any unit testing
framework in place, it does support the time-honored tradition of
debug-by-printf.%
%
\footnote{The ``f'' is not a typo --- it's a historical artifact of
  the name of the \code{printf} function used for formatted printing
  in C.}

Stan supports print statements with one or more string or expression
arguments.  Because Stan is an imperative language, variables can have
different values at different points in the execution of a program.
Print statements can be invaluable for debugging, especially for a
language like Stan with no stepwise debugger.

For instance, to print the value of variables \code{y} and
\code{z}, use the following statement.
%
\begin{stancode}
print("y=", y, " z=", z);
\end{stancode}
%
This print statement prints the string ``y='' followed by the value of
\code{y}, followed by the string `` z=''
(with the leading space), followed by the value of the variable
\code{z}.

Each print statement is followed by a new line.  The specific ASCII
character(s) generated to create a new line are platform specific.

Arbitrary expressions can be used.  For example, the statement
\begin{stancode}
print("1+1=", 1+1);
\end{stancode}
%
will print ``1 + 1 = 2'' followed by a new line.

Print statements may be used anywhere other statements may be used,
but their behavior in terms of frequency depends on how often the
block they are in is evaluated.



\section{Comments}\label{comments-programming.section}

\subsection{Code Never Lies}

The machine does what the code says, not what the documentation says.
Documentation, on the other hand, might not match the code.  Code
documentation easily rots as the code evolves if the documentation is
not well maintained.

Thus it is always preferable to write readable code as opposed to
documenting unreadable code.  Every time you write a piece of
documentation, ask yourself if there's a way to write the code in such
a way as to make the documentation unnecessary.


\subsection{Comment Styles in Stan}

Stan supports \Cpp-style comments with \code{//} for line comments and
\code{/*} and \code{*/} as block comment wrappers.  The recommended
style is to use line-based comments for short comments on the code or
to comment out one or more lines of code.  Bracketed comments are then
reserved for long documentation comments.  The reason for this
convention is that bracketed comments cannot be wrapped inside of
bracketed comments.

\subsection{What Not to Comment}

When commenting code, it is usually safe to assume that you are
writing the comments for other programmers who understand the basics
of the programming language in use.  In other words, don't comment the
obvious.  For instance, there is no need to have comments
such as the following, which add nothing to the code.
%
\begin{stancode}
y ~ normal(0, 1);  // y has a standard normal distribution
\end{stancode}
%
A Jacobian adjustment for a hand-coded transform might be worth
commenting, as in the following example.
%
\begin{stancode}
exp(y) ~ normal(0, 1);
// adjust for change of vars: y = log | d/dy exp(y) |
target += y;
\end{stancode}
%
It's an art form to empathize with a future code reader and decide
what they will or won't know (or remember) about statistics and Stan.

\subsection{What to Comment}

It can help to document variable declarations if variables are given
generic names like \code{N}, \code{mu}, and \code{sigma}.  For
example, some data variable declarations in an item-response model
might be usefully commented as follows.
%
\begin{stancode}
int<lower=1> N;   // number of observations
int<lower=1> I;   // number of students
int<lower=1> J;   // number of test questions
\end{stancode}
%
The alternative is to use longer names that do not require comments.
%
\begin{stancode}
int<lower=1> n_obs;
int<lower=1> n_students;
int<lower=1> n_questions;
\end{stancode}
%
Both styles are reasonable and which one to adopt is mostly a matter of
taste (mostly because sometimes models come with their own naming
conventions which should be followed so as not to confuse readers of
the code familiar with the statistical conventions).

Some code authors like big blocks of comments at the top explaining
the purpose of the model, who wrote it, copyright and licensing
information, and so on.  The following bracketed comment is an
example of a conventional style for large comment blocks.
%
\begin{stancode}
/*
 * Item-Response Theory PL3 Model
 * -----------------------------------------------------
 * Copyright: Joe Schmoe  <joe@schmoe.com>
 * Date:  19 September 2012
 * License: GPLv3
 */

data {
  // ...
\end{stancode}
%
The use of leading asterisks helps readers understand the scope of the
comment.  The problem with including dates or other volatile
information in comments is that they can easily get out of synch with
the reality of the code.  A misleading comment or one that is wrong is
worse than no comment at all!
